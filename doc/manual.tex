\documentclass[11pt, a4paper]{article}
\usepackage[left=1.4cm,text={18.2cm, 25.2cm},top=2.3cm]{geometry}

\usepackage{times}
\usepackage[czech]{babel}
\usepackage[IL2]{fontenc}
\usepackage[utf8]{inputenc}
\usepackage{listings}

\usepackage{amsmath, amsthm, amssymb}
\usepackage[bottom]{footmisc}
\usepackage{graphicx}
\usepackage{hyperref}

\usepackage{etoolbox}
\patchcmd{\thebibliography}{\section*{\refname}}{}{}{}

\lstset{
    language=C,
    basicstyle=\ttfamily,
    stepnumber=1,
    numbersep=5pt,
    showstringspaces=false,
    tabsize=4,
    breaklines=true,
    breakatwhitespace=false,
    frame=single,
    captionpos=b
}

\begin{document}

\begin{titlepage}
\thispagestyle{empty}

    \begin{center}

        {\Huge \textsc{Vysoké učení technické v~Brně \\[0.5em]}}

        {\huge \textsc{Fakulta informačních technologií}}

        \vspace{\stretch{0.382}}

        {\LARGE Sieťové aplikácie a správa sietí \\[0.5em]
        PCAP NetFlow v5 exportér
         }

        \vspace{\stretch{0.618}}

    \end{center}
{\Large 2024 \hfill Boris Hatala (xhatal02)}

\end{titlepage}

\newpage
\tableofcontents
\newpage

\section{Úvod}
Táto technická správa popisuje implementáciu PCAP NetFLow v5 exportéru vo forme konzolovej aplikácie
napísanej v jazyku C++. Aplikácia je kompilovaná pomocou g++ a je určená pre Linux.

\subsection*{Problematika}
\addcontentsline{toc}{subsection}{Problematika}
V rámci tohto projektu sa zameriavame na analýzu sieťovej prevádzky a jej prevod do formátu NetFlow v5, ktorý predstavuje štandard pre monitorovanie a analýzu sieťových tokov. Sieťová prevádzka je zvyčajne zachytávaná do PCAP súborov, ktoré obsahujú detailný záznam o jednotlivých sieťových paketoch. Z týchto záznamov je možné rekonštruovať a analyzovať toky, ktoré predstavujú agregované informácie o komunikácii medzi dvoma sieťovými bodmi, charakterizované napríklad IP adresami, portmi a protokolmi.
\vspace{0.3cm}

Cieľom projektu je vytvoriť nástroj \texttt{p2nprobe}, ktorý dokáže:
\begin{itemize}
    \item Načítat sieťové pakety zo súboru vo formáte PCAP
    \item Extrahovať informácie o sieťových tokoch výhradne pre protokol TCP
    \item Agregovať tieto informácie do formátu NetFlow v5
    \item Odosielať agregované toky na vzdialený kolektor prostredníctvom UDP
\end{itemize}

\section{Návod na použitie}
Program je potrebné preložiť pomocou prikázu \texttt{make}. Následne je možné ho spustiť z príkazového riadku s nasledovnými argumentami:
\begin{verbatim}
./p2nprobe [-a <akt. t.>] [-i <neak. t.>] <pcap súbor> <kolektor_ip:kolektor_port>
\end{verbatim}
\begin{itemize}
    \item \texttt{-a <aktívny timeout>} - Nastaví aktívny timeout v sekundách
    \item \texttt{-i <neaktívny timeout>} - Nastaví neaktívny timeout v sekundách
    \item \texttt{<pcap súbor>} - Cesta k PCAP súboru, ktorý sa má analyzovať.
    \item \texttt{<kolektor\_ip:kolektor\_port>} - Adresa a port NetFlow kolektora.
\end{itemize}

Argumenty \texttt{-a} a \texttt{-i} sú voliteľné, ak nepoužijete tieto argumenty, tak aktívny aj neaktívny timeout bude automaticky nastavený na 60 sekúnd. Argumenty v príkazovom riadku môžu byť v hocijakom poradí.
\subsection*{Príklad spustenia}
\addcontentsline{toc}{subsection}{Príklad spustenia}

\begin{verbatim}
./p2nprobe large.pcap 127.0.0.1:2055 -a 10 -i 3 
\end{verbatim}
Program spracuje súbor \texttt{large.pcap}, ktorého toky vo formáte NetFlow v5 pošle na kolektor s ip adresou \texttt{127.0.0.1} a portom \texttt{2055}. Aktívny timeout bude nastavený na 10 sekúnd a neaktívny na 3 sekundy.

\section{Návrh}
Aplikácia je rozdelená na zdrojový súbor \texttt{p2nprobe.cpp} a hlavičkový súbor \texttt{p2nprobe.h}. V tejto kapitole sú popísané jednotlivé časti implementácie a technické detaily.
\vspace{0.3cm}

\subsection*{Knižnice}
\addcontentsline{toc}{subsection}{Knižnice}

Program využíva knižnice:
\begin{itemize}
    \item \texttt{pcap} na spracovanie PCAP súborov.
    \item Siete (\texttt{netinet} a \texttt{arpa/inet}) na prácu s IP adresami a TCP/UDP hlavičkami.
    \item Štandardná knižnica C++ (\texttt{map}, \texttt{string}) na ukladanie aktívnych a čakajúcich tokov.
\end{itemize}

\subsection*{Štruktúry}
\addcontentsline{toc}{subsection}{Dôležité štruktúry}

\begin{description}
    \item[\texttt{Flow}] Štruktúra reprezentujúca jednotlivý tok, obsahuje:
    \begin{itemize}
        \item Zdrojová a cieľová IP adresa (\texttt{struct in\_addr}).
        \item Zdrojový a cieľový port (\texttt{uint16\_t}).
        \item Počet paketov a bajtov v toku (\texttt{uint32\_t}).
        \item Čas začiatku a konca toku (\texttt{uint32\_t}).
    \end{itemize}
    \item[\texttt{NetFlowV5Packet}] Štruktúra obsahujúca hlavičku pakety NetFlow v5 a maximálne 30 záznamov typu NetFlow v5.
    \\
    \begin{lstlisting}[language=C++]
    struct NetFlowV5Packet {
        struct NetFlowV5Header header;
        struct NetFlowV5Record records[30];
};
\end{lstlisting}
    \begin{itemize}
        \item \texttt{NetFlowV5Header} je štruktúra, ktorá predstavuje hlavičku NetFlow v5 pakety, nachádza sa v nej verzia pakety, doba prevádzky exportéra v milisekundách pred odoslaním, počet záznamov v pakete atď.
        \item \texttt{NetFlowV5Record} je štruktúra, ktorá obsahuje informácie o jednotlivých záznamoch v NetFlow v5 pakete ako sú zdrojová a cieľová ip adresa a port, začiatok a koniec záznamu v milisekundách vzhľadom na dobu prevádzky exportéra atď. 
    \end{itemize}

    \item[\texttt{activeFlows}] 
    Typ \texttt{std::map} obsahujúci aktívne toky, kde kľúčom je zdrojová a cieľová IP adresa a port.
    \item[\texttt{flowsBuffer}] Typ \texttt{std::map} obsahujúci toky, ktoré čakajú na export. 
\end{description}

\section{Implementácia}
Vstupným bodom programu je funkcia \texttt{main}, ktorá spracuje argumenty z príkazového riadka a 
následne sa pokúsi otvoriť PCAP súbor.

\subsection*{Logika spracovania paketov}
\addcontentsline{toc}{subsection}{Logika spracovania paketov}

Zavolaním funkcie \texttt{pcap\_loop} sa začne spracovávať každý paket pomocou callback 
funkcie \texttt{callback}, kde sa zisťuje, či je paket TCP.
Ak áno, tak sa z neho extrahujú informácie o toku a aktualizuje sa záznam v \texttt{activeFlows}.
Teda, buď sa vytvorí nový tok, alebo sa aktualizuje už existujúci. Toky sú ukladané do mapy 
\texttt{activeFlows} podľa zdrojovej a cieľovej IP adresy a portu.

\subsection*{Timeouty}
\addcontentsline{toc}{subsection}{Timeouty}

Po každom spracovanom pakete sa prejde mapa \texttt{activeFlows} a zistí sa, 
či nejaký tok nepresiahol timeout. Tento výpočet je vykonávaný na základe časového údaju 
z poslednej načítanej pakety. Ak tok presiahol aktívny timeout alebo neaktívny,
 tak sa presunie do mapy \texttt{flowsBuffer} na export.

\subsection*{Inicializácia NetFlow v5 pakety}
\addcontentsline{toc}{subsection}{Inicializácia NetFlow v5 pakety}

Inicializácia NetFlow v5 pakety sa vykonáva vo funkcii \texttt{initNetFlowV5Packet}. 
Táto funkcia nastaví všetky potrebné polia hlavičky NetFlow v5 pakety a zabezpečí, že paketa 
je pripravená na naplnenie záznamami o tokoch.
Funkcia \texttt{initNetFlowV5Packet} vynuluje celú štruktúru pakety pomocou \texttt{memset}, 
nastaví verziu pakety, vypočíta dobu prevádzky exportéra v milisekundách od začiatku programu a nastaví pole 
\texttt{sysUptime}, nastaví \texttt{unixSecs} a \texttt{unixNsecs}, a sekvenciu toku do poľa \texttt{flowSequence}.

\subsection*{NetFlow v5 záznam}
\addcontentsline{toc}{subsection}{NetFlow v5 záznam}

Záznamy sú postupne pridávané do NetFlow v5 pakety v cykle, kde sa prechádza mapa \texttt{flowsBuffer}.
Položka first a last sú vypočítané v milisekundách na základe časového údaja pakety a času začiatku programu.

\subsection*{Export}
\addcontentsline{toc}{subsection}{Export}

Po každom spracovanom pakete sa prejde mapa \texttt{flowsBuffer} a overí sa či nejaký tok čaká na export.
Ak áno, vytvárajú sa záznamy v NetFlow v5 formáte, zapĺňa sa paketa a záznamy z 
\texttt{flowsBuffer} sa postupne mažú. Paketa sa odošle v momente, keď je naplnená, 
teda obsahuje 30 záznamov, alebo keď program skončí spracovávanie PCAP súboru. V takom 
prípade sa všetky zvyšné toky exportujú. Po exportovaní pakety sa incializuje nová a 
proces sa opakuje.

\section{Testovanie}
Na testovanie boli použité  PCAP súbory generované pomocou programu \texttt{Wireshark} 
alebo \texttt{tcpdump}, kolektor \texttt{nfcapd} z nástroja \texttt{nfdump}, 
referenčný exportér \texttt{softwflowd} a program \texttt{Wireshark}. Pri testovaní a debugovaní
mi rovnako pomáhali aj rôzne print funckie.

\subsection*{Pomocou \texttt{nfcapd} a \texttt{softwflowd}}
\addcontentsline{toc}{subsection}{Pomocou \texttt{nfcapd} a \texttt{softwflowd}}

Kolektor \texttt{nfcapd} som spustil na lokálnej adrese a porte 2055.
Exportoval som rovnaké PCAP súbory pomocou \texttt{softwflowd} a \texttt{p2nprobe} na 
kolektor \texttt{nfcapd} a následne som pomocou nástroja \texttt{nfdump} porovnal výstupy.
Analyzoval som počet exportovaných tokov a správnosť exportovaných informácií.

\subsection*{Pomocou \texttt{Wireshark}}
\addcontentsline{toc}{subsection}{Pomocou \texttt{Wireshark}}

Po spustení kolektoru \texttt{nfcapd} som exportoval PCAP súbor pomocou \texttt{p2nprobe} a
\texttt{softwflowd} sledoval som správnosť exportovaných informácií pomocou programu 
\texttt{Wireshark}.

\newpage

\section*{Zdroje}
\addcontentsline{toc}{section}{Zdroje}

\begin{thebibliography}{9}

\bibitem{cisco_format} 
Cisco Systems: \emph{NetFlow Export Datagram Format}. Dostupné z: 
\url{https://www.cisco.com/c/en/us/td/docs/net_mgmt/netflow_collection_engine/3-6/user/guide/format.html#wp1006108}.

\bibitem{lucas_network_flow}
Lucas, M. W.: \emph{Network Flow Analysis}. No Starch Press, 2010.

\bibitem{rfc3954} 
Cisco Systems: \emph{RFC 3954: Cisco Systems NetFlow Services Export Version 9}. Dostupné z: 
\url{https://datatracker.ietf.org/doc/html/rfc3954}.

\bibitem{rfc7011} 
Claise, B., Trammell, B., Aitken, P.: \emph{RFC 7011: Specification of the IP Flow Information Export (IPFIX) Protocol for the Exchange of Flow Information}. Dostupné z: 
\url{https://datatracker.ietf.org/doc/html/rfc7011}.

\end{thebibliography}

\end{document}